\documentclass[11pt]{article}
\usepackage{setspace}

\usepackage{times,float, multicol}
\usepackage[pdftex]{graphicx}
\usepackage[rflt]{floatflt}
\usepackage{hyperref}

\setlength{\textwidth}{6.5in}
\setlength{\textheight}{9.0in}
\setlength{\topmargin}{-.5in}
\setlength{\oddsidemargin}{-.0600in}
\setlength{\evensidemargin}{.0625in}

\newcommand{\secref}[1]{Section~\ref{#1}}


\def\TreeSearch{{TreeSearch}}

\title{ADAGE on Grids User Guide}

\author{Derrick Stolee \\ 
	Iowa State University\\
	\texttt{dstolee@iastate.edu}
       }
       
\begin{document}

\maketitle
\vspace{-.3in}
\begin{abstract}
	The ADAGE framework is a method for automatically constructing discharging arguments.
	ADAGE on Grids is the specific implementation for bounding density of sets in grids.
	This document details the strategy of ADAGE on Grids along with implementation, compilation, and execution details.
\end{abstract}

\section{Introduction}
\label{sec:Introduction}


\subsection{Acquiring \TreeSearch}

The latest version of ADAGE on Grids and its documentation is publicly available on GitHub \cite{github} at the address \href{http://www.github.com/derrickstolee/ADAGE/}{http://www.github.com/derrickstolee/ADAGE/}.
It will also be available at \url{http://www.math.iastate.edu/dstolee/r/adage.htm}.

\section{Strategy}
\label{sec:Strategy}







\section{Data Structures}
\label{sec:structures}
	
\noindent{\bf \texttt{writeStatistics()}:}

	

\section{Compilation}
\label{sec:Compilation}

To compile ADAGE on Grids, run \texttt{make} in the \texttt{src-grids} directory.
Then, to compile the linear programming executables, run \texttt{make -f Makefile-GLPK} in the \texttt{src} directory.

more?

\section{Execution}
\label{sec:Execution}

list the three commands in order...





\end{document}
